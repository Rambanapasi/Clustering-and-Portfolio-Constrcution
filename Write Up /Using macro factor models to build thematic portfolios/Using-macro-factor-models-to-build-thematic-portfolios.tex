\documentclass[11pt,preprint, authoryear]{elsarticle}

\usepackage{lmodern}
%%%% My spacing
\usepackage{setspace}
\setstretch{1.2}
\DeclareMathSizes{12}{14}{10}{10}

% Wrap around which gives all figures included the [H] command, or places it "here". This can be tedious to code in Rmarkdown.
\usepackage{float}
\let\origfigure\figure
\let\endorigfigure\endfigure
\renewenvironment{figure}[1][2] {
    \expandafter\origfigure\expandafter[H]
} {
    \endorigfigure
}

\let\origtable\table
\let\endorigtable\endtable
\renewenvironment{table}[1][2] {
    \expandafter\origtable\expandafter[H]
} {
    \endorigtable
}


\usepackage{ifxetex,ifluatex}
\usepackage{fixltx2e} % provides \textsubscript
\ifnum 0\ifxetex 1\fi\ifluatex 1\fi=0 % if pdftex
  \usepackage[T1]{fontenc}
  \usepackage[utf8]{inputenc}
\else % if luatex or xelatex
  \ifxetex
    \usepackage{mathspec}
    \usepackage{xltxtra,xunicode}
  \else
    \usepackage{fontspec}
  \fi
  \defaultfontfeatures{Mapping=tex-text,Scale=MatchLowercase}
  \newcommand{\euro}{€}
\fi

\usepackage{amssymb, amsmath, amsthm, amsfonts}

\def\bibsection{\section*{References}} %%% Make "References" appear before bibliography


\usepackage[round]{natbib}

\usepackage{longtable}
\usepackage[margin=2.3cm,bottom=2cm,top=2.5cm, includefoot]{geometry}
\usepackage{fancyhdr}
\usepackage[bottom, hang, flushmargin]{footmisc}
\usepackage{graphicx}
\numberwithin{equation}{section}
\numberwithin{figure}{section}
\numberwithin{table}{section}
\setlength{\parindent}{0cm}
\setlength{\parskip}{1.3ex plus 0.5ex minus 0.3ex}
\usepackage{textcomp}
\renewcommand{\headrulewidth}{0.2pt}
\renewcommand{\footrulewidth}{0.3pt}

\usepackage{array}
\newcolumntype{x}[1]{>{\centering\arraybackslash\hspace{0pt}}p{#1}}

%%%%  Remove the "preprint submitted to" part. Don't worry about this either, it just looks better without it:
\makeatletter
\def\ps@pprintTitle{%
  \let\@oddhead\@empty
  \let\@evenhead\@empty
  \let\@oddfoot\@empty
  \let\@evenfoot\@oddfoot
}
\makeatother

 \def\tightlist{} % This allows for subbullets!

\usepackage{hyperref}
\hypersetup{breaklinks=true,
            bookmarks=true,
            colorlinks=true,
            citecolor=blue,
            urlcolor=blue,
            linkcolor=blue,
            pdfborder={0 0 0}}


% The following packages allow huxtable to work:
\usepackage{siunitx}
\usepackage{multirow}
\usepackage{hhline}
\usepackage{calc}
\usepackage{tabularx}
\usepackage{booktabs}
\usepackage{caption}


\newenvironment{columns}[1][]{}{}

\newenvironment{column}[1]{\begin{minipage}{#1}\ignorespaces}{%
\end{minipage}
\ifhmode\unskip\fi
\aftergroup\useignorespacesandallpars}

\def\useignorespacesandallpars#1\ignorespaces\fi{%
#1\fi\ignorespacesandallpars}

\makeatletter
\def\ignorespacesandallpars{%
  \@ifnextchar\par
    {\expandafter\ignorespacesandallpars\@gobble}%
    {}%
}
\makeatother

\newenvironment{CSLReferences}[2]{%
}

\urlstyle{same}  % don't use monospace font for urls
\setlength{\parindent}{0pt}
\setlength{\parskip}{6pt plus 2pt minus 1pt}
\setlength{\emergencystretch}{3em}  % prevent overfull lines
\setcounter{secnumdepth}{5}

%%% Use protect on footnotes to avoid problems with footnotes in titles
\let\rmarkdownfootnote\footnote%
\def\footnote{\protect\rmarkdownfootnote}
\IfFileExists{upquote.sty}{\usepackage{upquote}}{}

%%% Include extra packages specified by user
\usepackage{booktabs}
\usepackage{longtable}
\usepackage{array}
\usepackage{multirow}
\usepackage{wrapfig}
\usepackage{float}
\usepackage{colortbl}
\usepackage{pdflscape}
\usepackage{tabu}
\usepackage{threeparttable}
\usepackage{threeparttablex}
\usepackage[normalem]{ulem}
\usepackage{makecell}
\usepackage{xcolor}

%%% Hard setting column skips for reports - this ensures greater consistency and control over the length settings in the document.
%% page layout
%% paragraphs
\setlength{\baselineskip}{12pt plus 0pt minus 0pt}
\setlength{\parskip}{12pt plus 0pt minus 0pt}
\setlength{\parindent}{0pt plus 0pt minus 0pt}
%% floats
\setlength{\floatsep}{12pt plus 0 pt minus 0pt}
\setlength{\textfloatsep}{20pt plus 0pt minus 0pt}
\setlength{\intextsep}{14pt plus 0pt minus 0pt}
\setlength{\dbltextfloatsep}{20pt plus 0pt minus 0pt}
\setlength{\dblfloatsep}{14pt plus 0pt minus 0pt}
%% maths
\setlength{\abovedisplayskip}{12pt plus 0pt minus 0pt}
\setlength{\belowdisplayskip}{12pt plus 0pt minus 0pt}
%% lists
\setlength{\topsep}{10pt plus 0pt minus 0pt}
\setlength{\partopsep}{3pt plus 0pt minus 0pt}
\setlength{\itemsep}{5pt plus 0pt minus 0pt}
\setlength{\labelsep}{8mm plus 0mm minus 0mm}
\setlength{\parsep}{\the\parskip}
\setlength{\listparindent}{\the\parindent}
%% verbatim
\setlength{\fboxsep}{5pt plus 0pt minus 0pt}



\begin{document}



\begin{frontmatter}  %

\title{Guide to Using Partition Clustering to Construct Portfolios}

% Set to FALSE if wanting to remove title (for submission)




\author[Add1]{Gabriel Rambanapasi}
\ead{rambanapasi44@gmail.com}





\address[Add1]{Stellenbosch University, Stellenbosch, South Africa}

\cortext[cor]{Corresponding author: Gabriel Rambanapasi}


\vspace{1cm}


\begin{keyword}
\footnotesize{
K-Means\sep Clustering \sep Price Momentun \sep Volatility
\sep Diversification \\
\vspace{0.3cm}
}
\end{keyword}



\vspace{0.5cm}

\end{frontmatter}

\setcounter{footnote}{0}



%________________________
% Header and Footers
%%%%%%%%%%%%%%%%%%%%%%%%%%%%%%%%%
\pagestyle{fancy}
\chead{}
\rhead{}
\lfoot{}
\rfoot{\footnotesize Page \thepage}
\lhead{}
%\rfoot{\footnotesize Page \thepage } % "e.g. Page 2"
\cfoot{}

%\setlength\headheight{30pt}
%%%%%%%%%%%%%%%%%%%%%%%%%%%%%%%%%
%________________________

\headsep 35pt % So that header does not go over title




\hypertarget{introduction}{%
\section{\texorpdfstring{Introduction
\label{Introduction}}{Introduction }}\label{introduction}}

\begin{itemize}
\tightlist
\item
  aim to test momentum via cumulative return
\item
  use clusters to group high performers, low performers to each group.
\item
  build a portfolio, and check out returns.
\item
  compare the underlying thesis with what has happended
\end{itemize}

\hypertarget{clustering-and-appliactions-to-asset-management}{%
\section{Clustering and Appliactions to Asset
Management}\label{clustering-and-appliactions-to-asset-management}}

Unsupervised machine learning is a type of machine learning that
searches for patterns in datasets with no pre-existing labels and a
minimum of human intervention. One way in which unsupervised learning
can be applied in data science and other quantitive disciplines is
through clustering algorithms. Clustering is the process of grouping
objects based on similar characteristics. The algorithms designed to
cluster, achieve this function by connecting observation through
distances, density of data points, graphs, or various statistical
distributions. For a cluster to have meaning an algorithm has to
maximize intra-cluster similarity and minimize inter-cluster similarity,
such that each cluster contains information that's as dissimilar to
other
clusters(\protect\hyperlink{ref-kassambara2017practical}{Kassambara,
2017}). There exists various forms of cluster algorithms, each that
addresses a broader task of analysis. The algorithms can be divided into
two main types being partitioning clustering and hierarchical
clustering. The major difference between the divisions of clustering is
the partition clustering aims to specify a predetermined number of
clusters whilst does not
(\protect\hyperlink{ref-kassambara2017practical}{Kassambara, 2017}).
Within partition clustering, for data with a small set of variables,
K-means clustering and partitioning around medoids (PAM) are the most
frequently used due to their fast compuation and simplicity. With
K-means, each cluster is represented by the center or means of the data
points belonging to the entire dataset. This makes the algorithm
sensitive to outliers. However with PAM, each cluster is represented by
one of the objects in the cluster. The other partition clustering
algorithm used for datasets with a large number of variables is
Clustering Large Applications (CLARA).

In asset management, key to funds generating superior risk adjusted
returns is efficient portfolio diversification, thus presenting a great
application for partition clustering. Stocks would be separated into
groups through a clustering algorithm to maximize similarity within
groups and minimizes similarity between groups. Thus allowing managers
to select handpick stock to construct a diversified portfolio. Marvin
(\protect\hyperlink{ref-marvin2015creating}{2015}) use fundamental
ratios (turnover and profitability ratios) weighted equally and K-means
clustering to group US technology stocks listed on the NASDAQ and NYSE.
A diversified portfolio is then constructed based on within cluster
stock performance i.e.~stock selected are those that possess the highest
Sharpe ratio. Results over a period of 15 years that included the dot
com bubble and the global financial crises showed that cluster
portfolios exhibited more volatility than the benchmark (S\&P 500),
however returns to investors were above the benchmark at multiples
ranging from 3.5 to 5.7 times when earnings are reinvested into the
cluster portfolios. Bin (\protect\hyperlink{ref-bin2020k}{2020}) uses a
similar approach to Marvin
(\protect\hyperlink{ref-marvin2015creating}{2015}), however employing a
combination of market ratios and fundamental ratios (price to earnings
ratio, return on assets ratio and asset turnover ratio ). From this
study, compared to the S\&P 500, portfolios constructed using market
ratios under performed those that used fundamental ratios. \newpage

\hypertarget{data-and-methodology}{%
\section{Data and Methodology}\label{data-and-methodology}}

This section describes how we obtain the data set used in the study,
details the clustering process and validating metrics employed, to
obtain the results in \ref{Results}.

\hypertarget{obtaining-and-prepariing-the-dataset}{%
\subsection{Obtaining and prepariing the
dataset}\label{obtaining-and-prepariing-the-dataset}}

The data employed in this paper is based on the constituent list of the
Johannesburg All Share Index (ALSI) from January 1, 2000 to January 15
2024, which contains the list of the 164 companies with the highest
market value and liquidity. The historical price and volume data is
retrieved from Yahoo Finance and fundamental data from Bloomberg. From
historical price data obtained from Yahoo Finance, we filter stock that
have trading volume that exceed 1 000 000 shares traded per year and
exclude stock that have less than 90 percent of observations in the
historical price dataset.

To avoid large oscillations in the data, we transformed the price series
to include end of month data points thus returns calculations are based
on from the monthly data. Monthly historical prices are transformed
using simple returns and we assume that embedded in the price action are
cooperate events such as stock splits or consolidations of the shares.
Therefore there is no need to make additional transformations on the
return series to reflect corporate actions.

The measures of similarity used in this study are volatility and price
momentum. To cluster stock based on the two measures, we apply a
percentile ranking criterion on stock scores during a time period. For
price momentum, describes the causuality between relatively strong
performance and high future return and vice versa. Ranking highly
implies that strong performers and thus higher returns than weak
performers. This study defines cross sectional momentum as the trailing
6 month cumulative return Jegadeesh \& Titman
(\protect\hyperlink{ref-jegadeesh1993returns}{1993}). For volatility,
using a 12 month lockback period compute the standard deviation. The
results of the ranking are shown in Table \ref{tab1}

\hypertarget{stocks-clustering}{%
\subsection{Stocks clustering}\label{stocks-clustering}}

\hypertarget{lloyds-algorithm}{%
\subsection{Lloyd's algorithm}\label{lloyds-algorithm}}

We employ the K-means that partitions \(n\) observations into \(k\)
clusters ???. The goal is to minimize the within cluster sum of squares
or analytically:

\(\underset{s}{\arg \min } \sum_{i=1}^k \sum_{x \in S_i}\left\|x-\mu_i\right\|^2\)

where \(x\) are the observations, \(S=S_1, S_2, \ldots, S_k\) are the
sets of observations, and \(\mu_i\) is the mean of the points in
\(S_i\). To arrive at the optimum number of clusters, we utilize the
most popular algorithm called the Lloyd's algorithm that is closely
followed by Marvin (\protect\hyperlink{ref-marvin2015creating}{2015}),
Bin (\protect\hyperlink{ref-bin2020k}{2020}) \& Xu, Xu \& Wunsch
(\protect\hyperlink{ref-xu2010clustering}{2010}).

Analytically:

Given a set of points
\(\left\{x_1, \cdots x_n\right\}\left(x_i \in \mathbb{R}^m\right)\),

\begin{itemize}
\item
  Initialize the K clusters with \(\left\{C_1, \cdots C_K\right\}\) with
  centers
  \(\left\{m_1, m_2, \cdots m_K\right\}\left(m_i \in \mathbb{R}^m\right)\).
  The centres are picked using the silhoutte method discussed in
  \ref{sil}
\item
  For all points \(x_i(i \in\{1, \cdots n\})\), find the centre that
  closest based on a eucliean distance \(d\). Following this, assign
  \(x_i\) to the cluster corresponding to the closest centre.
\end{itemize}

\(x_i \in C_j\) if
\(d\left(x_i, m_j\right) \leq d\left(x_i, m_l\right) \quad(\forall l \in\{1, \cdots K\})(j \neq l)(\forall i \in\{1, \cdots n\})\).

\begin{itemize}
\tightlist
\item
  Recalculate the center for each cluster \(C_l(l \in\{1, \cdots K\})\).
  The new cluster centres are the mean of the points in the same
  cluster.
\end{itemize}

\(m_l=\frac{1}{\left|C_l\right|} \sum_{x_p \in C_l} x_p \quad(\forall l \in\{1, \cdots K\})\).

\begin{itemize}
\tightlist
\item
  Repeat processes two and three until no cluster has any change in
  point assignment.
\end{itemize}

\hypertarget{silhoutte-index}{%
\subsection{\texorpdfstring{Silhoutte index
\label{sil}}{Silhoutte index }}\label{silhoutte-index}}

To evaluate the goodness of fit of partitioning using K means clustering
the silhoutte index is used. A summary that closely follows Rousseeuw
(\protect\hyperlink{ref-rousseeuw1987silhouettes}{1987}), can be seen
below:

Given \(n\) data points \(\left\{x_1, \cdots x_n\right\}\), a
partitioning result of \(K\) cluster \(\left\{C_1, \cdots C_K\right\}\)
and distance metric \(d\), for each \(x_i\) in cluster \(C_l\), define

\(a\left(x_i\right)=\frac{1}{\left|C_l\right|-1} \sum_{\forall x_j \in C_l, i \neq j} d\left(x_i, x_j\right)\)

where \(a(x_i)\) is the mean dissimilarity between \(x_i\) to all other
points within the same cluster.

For each point \(x_i\) in cluster \(C_l\), define

\(b\left(x_i\right)=\min _{\forall p \in\{1, \cdots K\}, p \neq l} \frac{1}{\left|C_p\right|} \sum d\left(x_i, x_j\right.)\)

\(b(x_i)\) is the minimum dissimilarity between \(x_i\) and all points
in some \(C_p\) which does not contain \(x_i\).

For each point \(x_i\) in cluster \(C_l\), their silhoutte index is
defined as

\(s\left(x_i\right)= \begin{cases}1-\frac{a\left(x_i\right)}{b\left(x_i\right)} & \text { if } a\left(x_i\right)<b\left(x_i\right) \\ 0 & \text { if } a\left(x_i\right)=b\left(x_i\right) \\ \frac{b\left(x_i\right)}{a\left(x_i\right)}-1 & \text { if } a\left(x_i\right)>b\left(x_i\right)\end{cases}\)

where \(s(x_i)\) ranges between \([-1, 1]\).

For \(s(x_i)\) that approaches 1, it means that \(a(x_i)\) needs to be
significantly smaller than \(b(x_i)\), implying that within-cluster mean
dissimilarity is much less than the smallest between-cluster mean
dissimilarity, and thus the model does a good job clustering similar
points together.

For the \(s(x_i)\) that approaches 0, \(a(x_i)\) needs to be
significantly greater than \(b(x_i)\), implying that within-cluster mean
dissimilarity is much greater than the smallest between- cluster mean
dissimilarity, and thus the model does a poor job clustering similar
points together.

For this study, we choose \(K\) with the highest silhoutte index/value

\hypertarget{portfolio-backtest}{%
\subsection{Portfolio Backtest}\label{portfolio-backtest}}

The out-of-sample performance of cluster portfolios is compared to the
benchmark, the JSE All Share Index. To manage risk exposure to a single
asset or industry, we use a cap on each asset's allocation. Thus use a
single company methodology similar to Standard \& Poor Capping
Methodology (\protect\hyperlink{ref-sp}{S\&P, 2023}). In a single
company capping methodology, no company in an index (cluster in our
case) is allowed to breach a certain pre-determined weight as of each
rebalancing period \footnote{we set the maximum weight to be that of an
  equally weight cluster}. Theoretically, this should preserve the
within cluster diversification benefits and allow the portfolio value to
either increase or decrease depending on stock performance during the
quarter. We rebalance the portfolios once every three months, similar to
the frequency of rebalancing conducted by the JSE on the JSE/FTSE
indices, that is, re balance on the last daty of March, June, September
and December.

\hypertarget{results}{%
\section{\texorpdfstring{Results
\label{Results}}{Results }}\label{results}}

\newpage

\hypertarget{references}{%
\section{References}\label{references}}

\hypertarget{refs}{}
\begin{CSLReferences}{1}{0}
\leavevmode\vadjust pre{\hypertarget{ref-asness2011momentum}{}}%
Asness, C. 2011. Momentum in japan: The exception that proves the rule.
\emph{The Journal of Portfolio Management}. 37(4):67--75.

\leavevmode\vadjust pre{\hypertarget{ref-bin2020k}{}}%
Bin, S. 2020. K-means stock clustering analysis based on historical
price movements and financial ratios.

\leavevmode\vadjust pre{\hypertarget{ref-jegadeesh1993returns}{}}%
Jegadeesh, N. \& Titman, S. 1993. Returns to buying winners and selling
losers: Implications for stock market efficiency. \emph{The Journal of
finance}. 48(1):65--91.

\leavevmode\vadjust pre{\hypertarget{ref-kassambara2017practical}{}}%
Kassambara, A. 2017. \emph{Practical guide to cluster analysis in r:
Unsupervised machine learning}. Vol. 1. Sthda.

\leavevmode\vadjust pre{\hypertarget{ref-marvin2015creating}{}}%
Marvin, K. 2015. Creating diversified portfolios using cluster analysis.
\emph{Princeton University}.

\leavevmode\vadjust pre{\hypertarget{ref-rousseeuw1987silhouettes}{}}%
Rousseeuw, P.J. 1987. Silhouettes: A graphical aid to the interpretation
and validation of cluster analysis. \emph{Journal of computational and
applied mathematics}. 20:53--65.

\leavevmode\vadjust pre{\hypertarget{ref-sp}{}}%
S\&P. 2023. \emph{Index mathematics methodology}.

\leavevmode\vadjust pre{\hypertarget{ref-xu2010clustering}{}}%
Xu, R., Xu, J. \& Wunsch, D.C. 2010. Clustering with differential
evolution particle swarm optimization. In IEEE \emph{IEEE congress on
evolutionary computation}. 1--8.

\end{CSLReferences}

\hypertarget{appendix}{%
\section{Appendix}\label{appendix}}

\hypertarget{appendix-a}{%
\subsection{Appendix A}\label{appendix-a}}

\begingroup\fontsize{12pt}{13pt}\selectfont
\begin{longtable}{lrr}
  \toprule
Ticker & Volatility  Rank & Price Momentum Rank \\ 
  \hline 
\endhead 
\hline 
{\footnotesize Continued on next page} 
\endfoot 
\endlastfoot 
 \midrule
ABG & 26.67 & 21.67 \\ 
  ANG & 16.67 & 16.67 \\ 
  BIK & 80.00 & 73.33 \\ 
  EQU & 90.00 & 86.67 \\ 
  FFA & 3.33 & 53.33 \\ 
  GFI & 50.00 & 30.00 \\ 
  GLN & 56.67 & 36.67 \\ 
  HAR & 46.67 & 100.00 \\ 
  HMN & 66.67 & 60.00 \\ 
  KAP & 70.00 & 33.33 \\ 
  KBO & 100.00 & 3.33 \\ 
  MCG & 73.33 & 43.33 \\ 
  MRP & 10.00 & 46.67 \\ 
  MTM & 60.00 & 76.67 \\ 
  NPH & 96.67 & 26.67 \\ 
  ORN & 30.00 & 10.00 \\ 
  OUT & 20.00 & 56.67 \\ 
  PAN & 33.33 & 83.33 \\ 
  PIK & 36.67 & 6.67 \\ 
  PPC & 83.33 & 95.00 \\ 
  QLT & 53.33 & 90.00 \\ 
  RMH & 86.67 & 95.00 \\ 
  SAP & 43.33 & 80.00 \\ 
  SBK & 13.33 & 70.00 \\ 
  SOL & 40.00 & 13.33 \\ 
  TCP & 93.33 & 63.33 \\ 
  TFG & 23.33 & 40.00 \\ 
  TRU & 6.67 & 50.00 \\ 
  VKE & 76.67 & 66.67 \\ 
  WHL & 63.33 & 21.67 \\ 
   \bottomrule
\caption{Clustering Similarity Measures\label{tab1}} 
\end{longtable}
\endgroup

\hypertarget{appendix-b}{%
\subsection{Appendix B}\label{appendix-b}}

\bibliography{Tex/ref}





\end{document}
